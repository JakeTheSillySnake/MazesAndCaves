%% LyX 2.3.6 created this file.  For more info, see http://www.lyx.org/.
%% Do not edit unless you really know what you are doing.
\documentclass[english]{article}
\usepackage[T1]{fontenc}
\usepackage[latin9]{inputenc}
\usepackage{color}

\makeatletter
%%%%%%%%%%%%%%%%%%%%%%%%%%%%%% User specified LaTeX commands.
\usepackage{babel}
















\makeatother

\usepackage{babel}
\begin{document}
\title{\texttt{\textbf{Mazes \& Caves Manual}}}
\author{by JakeTheSillySnake (@ginzburg\_jake on Telegram)}
\date{Last updated March 2025}

\maketitle
\tableofcontents{}

\section{Introduction}

This document describes how to install, run, check and remove Maze
on Linux and Unix-like systems. Please note that the program was written
and tested on Ubuntu 22.04 LTS, so its behaviour may differ if you
use other versions or OS distributions. It should work on Windows
if you build with MinGW and qmake, but the UI may be wonky.

\section{Installation \& Running}

\subsection{Prerequisites}

Correct compilation and running of the program depends on other utilities
and libraries. Check that you have their latest versions before proceeding: 
\begin{description}
\item [{App~Compilation~\&~Running}] gcc, make, Qt5+ 
\item [{Testing}] GTest library 
\item [{GCOV~Report}] gcov, lcov 
\item [{Leaks~Check}] valgrind 
\item [{Convert~Manual~to~DVI}] texi2dvi 
\end{description}

\subsection{Setup}

Download or clone (\texttt{git clone <link\_to\_repo>}) the source
repository to where you can easily find it. Then type and run the
following commands in the terminal: 
\begin{enumerate}
\item \texttt{cd <path-to-git-folder>/src/maze} 
\item \texttt{make install} 
\end{enumerate}
Now the program is compiled, placing the app in a folder named \texttt{src/build/}.
The app should open automatically. If you want to open it later using
command line, run\texttt{ make maze.} If there are errors, you're
likely missing some packages. Check \textbf{Prerequisites}.

\subsection{Formatting}

The maze is stored in a file as a number of rows and columns, as well
as two matrices containing the positions of vertical and horizontal
walls respectively. The first matrix shows the wall to the right of
each cell, and the second --- the wall at the bottom. An example
of such a file:

~

\textcolor{black}{4 4 }

\textcolor{black}{0 0 0 1 }

\textcolor{black}{1 0 1 1 }

\textcolor{black}{0 1 0 1 }

\textcolor{black}{0 0 0 1}

\textcolor{black}{~}

\textcolor{black}{1 0 1 0 }

\textcolor{black}{0 0 1 0 }

\textcolor{black}{1 1 0 1 }

\textcolor{black}{1 1 1 1}

~

A cave that has passed 0 simulation steps (only initialized) is stored
as a number of rows and columns, as well as a matrix containing the
positions of 'alive' and 'dead' cells. For example:

~

4 4 

0 1 0 1 

1 0 0 1 

0 1 0 0 

0 0 1 1

~

\subsection{Usage}

The app layout should be fairly intuitive. On the first load, you'll
see a horizontal \textbf{Menu Tab} plus an \textbf{Action Tab} on
the right side of the screen. The \textbf{Menu Tab->File} has the
following options: 
\begin{enumerate}
\item Upload~Maze --- \emph{choose .txt maze file to display }
\item Upload~Cave --- \emph{choose .txt cave file to display} 
\item Save~Maze --- \emph{save current maze as .txt file in the format
described above}
\item Exit --- \emph{close the app} 
\end{enumerate}
The \textbf{Action Tab }allows you to switch between \textbf{Maze}
and \textbf{Cave}, which changes the options currently available:
\begin{enumerate}
\item \textbf{Maze}
\begin{enumerate}
\item Rows \& Cols Boxes --- \emph{dimensions for a maze to be generated
(1 to 50)}
\item Generate Button --- \emph{generate and display a 'perfect' maze}\footnote{i.e. maze without inaccessible areas or loops, where only one path
exists between any two points}\emph{ using Eller's algorithm}
\item Save Button --- \emph{save current maze as .txt file in the format
described above}
\item Start \& End Boxes --- \emph{position of start \& end points of a
maze solution (row, column)}
\item Solve Button --- \emph{solve current maze}
\end{enumerate}
\item \textbf{Cave}
\begin{enumerate}
\item Rows \& Cols Boxes --- \emph{dimensions for a cave to be generated
(1 to 50)}
\item Cell Initialisation (\%) --- \emph{chance of individual cell in the
first generation being initialised (1-99\%; 50\% recommended)}
\item Generate Button --- \emph{generate and display initialised cave}
\item Next Step --- \emph{generate next step for current cave using cellular
automation (disabled when }\textbf{\textsl{\emph{Automode}}}\emph{
is on)}
\item Birth Limit --- \emph{lower limit of 'alive' neighbours a 'dead'
cell needs to be 'born' (0-7; 5 recommended)}
\item Death Limit --- \emph{lower limit of 'alive' neighbours 'alive' cell
needs to stay 'alive' (0-7; 4 recommended)}
\item Automode Button --- \emph{enables automatic next step generation}
\item Hold Frame For (ms) --- \emph{generate next step every N ms (40-2000,
200 recommended; disabled when }\textbf{\textsl{\emph{Automode}}}\emph{
is off)}
\end{enumerate}
\end{enumerate}

\subsection{Web UI}

Additionally, you can use the app without installing it, through a
local server. The web version of UI was built using WebAssembly and
Emscripten Compiler. To open it, go to the \texttt{src/maze/web\_ui}
folder and then:
\begin{itemize}
\item Open \texttt{maze.html} using Live Server extension in VS Code, OR
\item Run \texttt{\textcolor{black}{python}}\footnote{\texttt{\textcolor{black}{python3 in newer Unix-versions, or py on
Windows}}}\texttt{\textcolor{black}{{} -m http.server}} inside the folder (you
need to have Python installed)
\end{itemize}
Then go to \texttt{\textcolor{black}{localhost/maze.html}} address
in your preferred browser. Done!

\section{Structure \& testing}

The program was made using C++17 language and standard libraries,
with the interface provided by Qt. The source code can be found in
\texttt{maze/backend} and \texttt{maze/frontend} folders. The backend
modules can be tested with GTest: 
\begin{enumerate}
\item To run tests:\texttt{ make test} 
\item To display test coverage:\texttt{ make gcov\_report} 
\item To check for leaks: \texttt{make valgrind } 
\end{enumerate}
Running \texttt{make} or \texttt{make all} will reinstall and compile
the program. You can get DVI documentation with \texttt{make dvi}
or a distribution .tar.gz package with \texttt{make dist}. Also note: 
\begin{enumerate}
\item Not every maze and cave can be displayed by the app. Check \textbf{Formatting}.
\item Example files can be found in \texttt{src/assets/example\_files}.
\end{enumerate}

\section{Deinstallation}

Simply run \texttt{make uninstall.} This will remove the \texttt{build/}
directory but not the original download, which can be safely deleted
afterwards.

\subparagraph{If you wish to suggest an improvement or report a bug, contact me
@ginzburg\_jake (Telegram) or JakeTheSillySnake (GitHub).}
\end{document}
